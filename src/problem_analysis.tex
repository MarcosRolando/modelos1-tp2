El problema del enunciado nos describe una situación en la cual contamos con una cantidad
limitada de equipos de distintos tipos que componen diferentes batallones, queriendo
conseguir la mejor distribución posible de los mismos en las distintas ubicaciones
a defender contra los incendios. Además, cada batallón cuenta con experiencia únicamente
en ciertas ubicaciones, lo que limita las opciones de distribución de los mismos.

No todas las ubicaciones son equivalentes. Algunas tienen mayor prioridad que otras,
viendose reflejado en un puntaje o calificación mayor. El corazón del problema radica entonces
en cómo conseguir maximizar la calificación total de las ubicaciones defendidas, teniendo en cuentas
las limitaciones de recursos y experiencia de los batallones.

A priori, el problema no especifica claramente como se procesan las calificaciones de las ubicaciones
ni los recursos de los batallones. Por lo tanto, se asume que la calificación de cada ubicación
sólo puede ser contabilizada una única vez en caso de que el acumulado de equipos de los batallones
asignados a la ubicación cumpla con los requerimientos de esta. En caso de haya un excedente de equipos
en la ubicación, la calificación provista por la misma no se modificaría.

Analizando la situación resulta evidente que se trata de un problema de optimización combinatoria.
En particular, se trata de un problema de optimización de asignación, en el cual se busca asignar
los elementos de un conjunto a otro de manera que se maximice una función objetivo. Deberemos entonces
determinar qué batallones se formarán y serán asignados a qué ubicaciones, individualizando así cada posible combinación
de batallón y ubicación. Luego, deberemos evaluar cada una de estas combinaciones y quedarnos aquellas
que maximicen la función objetivo, es decir, la calificación total.

